\section{Einfaches Beispiel}

\begin{frame}
    \frametitle{Einfaches Beispiel}

    Matrix $A$ und Vektor $\vec{b}$:
    \begin{columns}[c]
        \begin{column}{0.5\hsize}\centering
            $$A = \begin{pmatrix} 1 & -\frac{1}{3}\\ -\frac{1}{3} & 1\\ \end{pmatrix}$$
        \end{column}

        \begin{column}{0.5\hsize}
            $$\vec{b} = \begin{pmatrix} 0 \\ 1\\ \end{pmatrix}$$
        \end{column}
    \end{columns}
 
    \hfil

    \hfil

    Klassische Lösung
    \begin{columns}[c]
        \begin{column}{0.5\hsize}
            \centering
            $$\vec{x} = \begin{pmatrix} \frac{3}{8}\\ \frac{9}{8}\\ \end{pmatrix}$$
        \end{column}

        \begin{column}{0.5\hsize}
            \centering
            Verhältnis der Lösung:
            $$\frac{ |x_0|^2}{ |x_1|^2}= \frac{\frac{9}{64}}{\frac{81}{64}} = \frac{1}{9}$$
        \end{column}
    \end{columns}
 
    \hfil

    \hfil

\end{frame}

\begin{frame}
    \frametitle{Einfach Beispiel}


    Eigenvektoren von $A$ sind:    
    \begin{columns}[c]
        \begin{column}{0.5\hsize}\centering
            $$ \vec{u_0} = \begin{pmatrix} \frac{-1}{\sqrt{2}}\\ \frac{-1}{\sqrt{2}}\\ \end{pmatrix}$$
        \end{column}
        \begin{column}{0.5\hsize}
            $$\vec{u_1} = \begin{pmatrix} \frac{-1}{\sqrt{2}}\\ \frac{1}{\sqrt{2}}\\ \end{pmatrix}$$
        \end{column}
    \end{columns}

    \hfil

    \hfil

    Enkodiert
    \begin{columns}[c]
            \begin{column}{0.5\hsize}\centering
                $$ \ket{u_0} = \frac{-1}{\sqrt{2}}\ket{0} + \frac{-1}{\sqrt{2}}\ket{1}$$
            \end{column}
            \begin{column}{0.5\hsize}
                $$\ket{u_1}= \frac{-1}{\sqrt{2}}\ket{0} + \frac{1}{\sqrt{2}}\ket{1}$$
            \end{column}
        \end{columns}
    \end{frame}

\begin{frame}
    \frametitle{Einfach Beispiel}

    Eigenvektoren von $A$ sind:    
    \begin{columns}[c]
        \begin{column}{0.5\hsize}\centering
            $$\lambda_0 = \frac{2}{3}$$
        \end{column}

        \begin{column}{0.5\hsize}
            $$\lambda_1 = \frac{4}{3}$$
        \end{column}
    \end{columns}

    \hfil

    \hfil

    Einkodiert:
    \begin{columns}[c]
        \begin{column}{0.5\hsize}\centering
            $$\ket{\widetilde{\lambda_0}} = \ket{01}$$
        \end{column}

        \begin{column}{0.5\hsize}
            $$\ket{\widetilde{\lambda_1}} = \ket{10}$$
        \end{column}
    \end{columns}



    \hfil

\end{frame}


\begin{frame}
    \frametitle{State Preparation}

    \begin{center}
    \includegraphics[width=10.5cm]{img/example_circuit/example_circuit.jpg}
    \end{center}

    \begin{enumerate}
        \item Anzahl Qubit for a-register: 1
        \item Anzahl Qubits für das c-Register: N = 2
        \item Anzahl Qubits für $\vec{b}$: $n_b = log_2(N) = log_2 (2) = 1$ 
    \end{enumerate}

\end{frame}

\begin{frame}
    \frametitle{State Preparation}

    \begin{itemize}
        \item $\vec{b}$ wird als Quantenzustand $\ket{b}$ kodiert
        \item in unserem Fall ist es sehr einfach
    \end{itemize}

    $$\vec{b} = \begin{pmatrix} 0\\ 1\\ \end{pmatrix} \Leftrightarrow \ket{b} = 0 \ket{0} + 1 \ket{1} = \ket{1}$$

\end{frame}

\begin{frame}
    \frametitle{State Preparation}

    \begin{center}
    \includegraphics[width=10.5cm]{img/example_circuit/example_circuit_1.jpg}
    \end{center}

    Wir starten im 1 Zustand
    $$\ket{\Psi_1} = \ket{1}_b\ \ket{00}_c\ \ket{0}_a = \ket{1000}$$

\end{frame}




\begin{frame}
    \frametitle{Quantum Phase Estimation}
    
    Wir führen QPE aus:
    $$\ket{\Psi_2} = \ket{b}_b \ket{\widetilde{\lambda_j}}_c\ket{0}_a
     =\sum_{j=0}^{2^{1}-1} b_j \ket{u_j} \ket{ \widetilde{\lambda}_j} \ket{0}$$

    $$=\left(-\frac{1}{\sqrt{2}} \ket{u_0} \ket{01} +\frac{1}{\sqrt{2}}  
    \ket{u_1} \ket{10} \right)  \ket{0}$$

    \hfil

    \hfil

    \begin{itemize}
    \item b-register: Zustand $\ket{b}$ in Eigenbasis von A: $ \ket{u_0}$ or $\ket{u_1}$
    \item jeweilige Koeffizienten: $b_0 =\frac{-1}{\sqrt{2}}$ and  $b_1 =\frac{1}{\sqrt{2}}$
    \item c-register: Eigenwerte $\ket{\widetilde{\lambda}_0}$ und $\ket{\widetilde{\lambda}_1}$ enkodiert als  $\ket{01}$ und $\ket{10}$
    \item a-register: ancilla Qubit $\ket{0}$ 
    \end{itemize}
\end{frame}

\begin{frame}
    \frametitle{Quantum Phase Estimation}
    \begin{center}
    \includegraphics[width=10.5cm]{img/example_circuit/example_circuit_2.jpg}
    \end{center}

    Wir erhalten:
    $$\ket{\Psi_2} =\left(-\frac{1}{\sqrt{2}} \ket{u_0} \ket{01} +\frac{1}{\sqrt{2}}  
    \ket{u_1} \ket{10} \right)  \ket{0}_a$$

\end{frame}

\begin{frame}
    \frametitle{Ancilla Roation - Eigenwerte invertieren}
    Wir invertieren das Ancilla Qubit:
    $$\sum_{j=0}^{2^{1}-1} b_j \ket{u_j} \ket{\widetilde{\lambda}_j} \left(\sqrt{1-\frac{C^2}{\widetilde{\lambda}_j^2}}\ket{0} + \frac{C}{\widetilde{\lambda}_j} \ket{1}\right)$$
    %$$=-\frac{1}{\sqrt{2}} \ket{u_0} \ket{01}\left(\sqrt{1-\frac{1}{1^2}}\ket{0} + \frac{1}{1} \ket{1}\right) +\frac{1}{\sqrt{2}}  \ket{u_1} \ket{10} \left(\sqrt{1-\frac{1}{2^2}}\ket{0} + \frac{1}{2} \ket{1}\right)$$
    $$=\left(-\frac{1}{\sqrt{2}} \ket{u_0} \ket{01}\left(\ket{0} +\ket{1}\right)+\frac{1}{\sqrt{2}} \ket{u_1} \ket{10}\right)\left(\sqrt{1-\frac{1}{4}}\ \ket{0} + \frac{1}{2} \ket{1}\right)$$

    \hfil

    \hfil

    Wir gehen davon aus, dass wir $\ket{1}$ messen.
    $$  =\sqrt{\frac{8}{5}}\left(-\frac{1}{\sqrt{2}} \ket{u_0}\ket{01}\ket{1} +\frac{1}{2\sqrt{2}} \ket{u_1} \ket{10}\right)\ket{1}_a$$
\end{frame}

\begin{frame}
    \frametitle{Ancilla Roation - Eigenwerte invertieren}
    \begin{center}

    \includegraphics[width=10.5cm]{img/example_circuit/example_circuit_3.jpg}
    \end{center}

    $$ \ket{\Psi_3} =\sqrt{\frac{8}{5}}\left(-\frac{1}{\sqrt{2}} \ket{u_0}\ket{01}\ket{1} +\frac{1}{2\sqrt{2}} \ket{u_1} \ket{10}\right)\ket{1}_a$$
\end{frame}



\begin{frame}
    \frametitle{Inverse Quantum Phase Estimation}
    Wir führen IQPE aus:
    $$\ket{x}_b \ket{00}_c \ket{1}_a $$
    
    $$ \ket{x}_b =  A^{-1} \ket{b} = 
    \sum_{i=0}^{2^{1}-1} 
    \lambda_i^{-1} b_i\ket{u_i} $$

    $$=\lambda_0^{-1} b_0\  \ket{u_0} +  \lambda_1^{-1} b_1\ket{u_1}$$
    $$=-\frac{1}{\frac{2}{3}\sqrt{2}} \ket{u_0} +\frac{1}   {\frac{4}{3}\sqrt{2}}  \ket{u_1}$$
    $$=\frac{2}{3}\sqrt{\frac{8}{5}} \left( -\frac{1}{\frac{2}{3}\sqrt{2}} \ket{u_0} +\frac{1}   {\frac{4}{3}\sqrt{2}}  \ket{u_1}\right)  \ket{00}_b \ket{1}_a $$

\end{frame}

\begin{frame}
    \frametitle{Inverse Quantum Phase Estimation}

Wegen Normalisierung der Eigenvektoren können wir 

\begin{itemize}
    \item $ \ket{u_0} = \frac{-1}{\sqrt{2}}\ket{0} + \frac{-1}{\sqrt{2}}\ket{1}$ 
    \item $ \ket{u_1}= \frac{-1}{\sqrt{2}}\ket{0} + \frac{1}{\sqrt{2}}\ket{1}$ 
\end{itemize}

$$ \ket{\Psi_4}= \frac{1}{2}\sqrt{\frac{2}{5}} \left( \ket{0} + 3\ket{1} \right)  \ket{00}_b \ket{1}_a $$

$$ \ket{\Psi_4}= \left(\frac{1}{2}\sqrt{\frac{2}{5}} \ket{0} +\frac{1}{2}\sqrt{\frac{2}{5}} * 3 \ket{1} \right) \ket{00}_b \ket{1}_a $$

\end{frame}

\begin{frame}
    \frametitle{Ancilla Roation - Eigenwerte invertieren}
    \begin{center}
    \includegraphics[width=10.5cm]{img/example_circuit/example_circuit_4.jpg}
    \end{center}

$$ \ket{\Psi_4}= \left(\frac{1}{2}\sqrt{\frac{2}{5}} \ket{0} +\frac{1}{2}\sqrt{\frac{2}{5}} * 3 \ket{1} \right) \ket{00}_b \ket{1}_a$$
\end{frame}


\begin{frame}
    \frametitle{Measurment}

Um die Wahrscheinlichkeit von $ \ket{u_0}$ und $\ket{u_1}$ zu erhalten, müssen wir ihre Koeffizienten quadrieren

$$ c_0=\left|\frac{1}{2}\sqrt{\frac{2}{5}}*1\right|^2 = \frac{1}{20} $$

$$ c_1=\left|\frac{1}{2}\sqrt{\frac{2}{5}}*3\right|^2 = \frac{9}{20} $$

Das Verhältnis im b-Register ist wie erwartet $1:9$.

\end{frame}



\begin{frame}
    \frametitle{Gesamte Rechnung}
    \begin{center}
    \includegraphics[width=10.5cm]{img/example_circuit/example_circuit_5.jpg}
    \end{center}


\end{frame}



