\section{Evaluierung}

\subsubsection{Laufzeit}
\begin{frame}
    \frametitle{Laufzeit}

    \textbf{Gauß Verfahren}

    $$ \mathcal{O}( N^3 ) $$

    \begin{itemize}
        \item nicht der schnellste Algorithmus
        \item gleiche constraints sind zu beachten!! 
    \end{itemize}

\end{frame}


\subsubsection{Laufzeit}
\begin{frame}
    \frametitle{Laufzeit}
    \begin{columns}[c]
        \begin{column}{0.5\hsize}
            \textbf{Classical}

            \hfil

            Conjugate gradient descent 

            $$ \mathcal{O}(\kappa s log{\left(\frac 1 \epsilon\right)} N ) $$


            \hfil

            \begin{itemize}
            \item $N :=$ is number of variables in linear system
            \item $\kappa= \frac {\lambda_{max}} {\lambda_{min}}$:  condition number
            \end{itemize}

        \end{column}
        

        \begin{column}{0.5\hsize}
            \textbf{Quanten Version}

            \hfil

            HHL

            $$ \mathcal{O}(\frac{\kappa^2s^2}{\epsilon}logN) $$

            \hfil

            \begin{itemize}
            \item $\epsilon$ := is the accuracy                                                            
            \item $s$ := is s-sparse matrix: each row has at most s nonzero entries
            \end{itemize}

        \end{column}
    \end{columns}
 
    \hfil



\end{frame}


\begin{frame}
    \frametitle{Laufzeit}
    \begin{columns}[c]
        \begin{column}{0.5\hsize}
            \textbf{Classical}

            \hfil

            Conjugate gradient descent 

            $$ \mathcal{O}(\kappa s log{\left(\frac 1 \epsilon\right)} N ) $$

            $$ \Rightarrow \mathcal{O} (N) $$

        \end{column}
        

        \begin{column}{0.5\hsize}
            \textbf{Quanten Version}

            \hfil

            HHL

            $$ \mathcal{O}(\frac{\kappa^2s^2}{\epsilon}logN) $$

            $$ \Rightarrow \mathcal{O} (log(N)) $$
        \end{column}

    \end{columns}
 
    \hfil

    \hfil

    \textbf{Takeaway}
    \begin{itemize}
        \item exponentialer speed up $\mathcal{O} (N)$ vs $\mathcal{O} (log(N))$
        \item klassischer algorithmus hat bessere Fehlerabhängigkeit: \\$log(\frac1{\epsilon})$ vs $\frac{1}{\epsilon}$
    \end{itemize}


\end{frame}



\subsubsection{Einschränkungen}
\begin{frame}
    \frametitle{Einschränkungen}
    \begin{enumerate}
        \item einfache Zustandsvorbereitung des Vektors $\vec b$ zum Quantenzustand $\ket{b}$

        \hfil

        \item niedrige condition number $\kappa$ 

        \hfil


        \item $A$ muss s-sparse sein

        \hfil


        \item nicht jeder Eintrag von $\ket{x}$ auslesbar

        \hfil


        \item Der Ressourcenbedarf sehr hoch


    \end{enumerate}

\end{frame}


\begin{frame}
    \frametitle{Einschränkungen}
    \begin{enumerate}
        \item niedrige condition number (es ist außerdem nicht einfach $\kappa$ im vorhinein zu ermitteln)

        \item muss s-sparse sein

        \item einfache Zustandsvorbereitung des Vektors $\vec b$ zum Quantenzustand $\ket{b}$
            \begin{itemize}
                \item wenn man $\ket{b}$ klassisch lesen/schreiben muss, ist der Geschwindigkeitsgewinn weg, da $\ket{b}$ $N$ Einträge hat→ qram
            \end{itemize}
        \item nicht jeder Eintrag von $\ket{x}$ auslesbar
            \begin{itemize}
                \item Nachbearbeitung muss erfolgen
                \item nur $log_2(n)$ Qubits -> nur eine Näherung
                \item statistische Informationen möglich (Verhältnis, Bereiche großer Einträge, ...)
            \end{itemize}
 
        \item Der Ressourcenbedarf sehr hoch
            \begin{itemize}
                \item Shors Algorithmus ist dem HHL-Algorithmus sehr ähnlich (aufgrund von QPE)
                \item untere Grenze von 4000 logischen Qubits (2048bit RSA)
                \item d.h. millionen physikalischer Qubits (für Fehlerkorrektur)
            \end{itemize}
    \end{enumerate}

\end{frame}

