\section{Zukunftsperspektiven}


\subsubsection{Anwendungen}
\begin{frame}
    \frametitle{Anwendungen}

    Hauptproblem
    \begin{itemize}
        \item Hauptproblem: gibt keinen vollständigen Vektor aus
        \item Aber einige Probleme können mit dieser Methode gelöst werden:
    \end{itemize}
    
\end{frame}

\begin{frame}
    \frametitle{Anwendungen}

    Machine Learning: Least-Square-Fitting
    \begin{itemize}
        \item Datenanpassung mit Least Square Fitting
        \item durch Berechnung einer Schätzung der inversen Matrix
    \end{itemize}

   \hfil


    Analysis of Large Sparse Electrical Networks 
    \begin{itemize}
        \item Elektrizitätsnetz vielen verbundenen Komponenten 
        \item geringe Anzahl Verbindungen zwischen den Komponenten
        \item Berechnung des Widerstands durch approximation von Erwartungswerten
    \end{itemize} 

   \hfil
   
    Es wäre wichtig, mehr Anwendungen zu finden, welche den Anforderungen entsprechen. 
\end{frame}

\begin{frame}
    \frametitle{Anwendung in IT-Security}

    HHL in der IT-Security
    \begin{itemize}
        \item in erster Linie nur für Lösen von linearen Systemen
        \item nicht direkt mit IT-Security verbunden
        \item aber Potenzial als Subroutine angewendet zu werden
    \end{itemize}

    \hfil

    Mögliche Anwendungen
     \begin{itemize}
        \item secure multi-party computation 
        \item zero-knowledge proofs
        \item cryptographic key generation and management
        \item big data analysis/pattern recognition (für Betrugserkennung)

    \end{itemize}
\   
\end{frame}

\subsubsection{Variationen}
\begin{frame}
    \frametitle{Variationen}

    Modifikationen und Optimierung
    \begin{itemize}
        \item QRAM zur Vorbereitung von $\ket{b}$
        \item kein Ancilla-Bit erforderlich unter bestimmten Voraussetzungen 
        \item Variable time amplitude amplification um condition number $\kappa$ zu verbessern
    \end{itemize}
    
\end{frame}

\subsubsection{Perspektive}
\begin{frame}
    \frametitle{Perspektive}

    \begin{itemize}
        \item  Großer Einfluss im Bereich Quantum Machine Learning 
        \item  noch keine bahnbrechenden Anwendungen (wie z.B. Shors Algorithmus zum Brechen von RSA)
        \item  aber viel aktive Forschung um neue Verbesserungen im Algorithmus zu finden
        \item  zeigt deutlichen Fortschritt in der Quantencomputing Welt
    \end{itemize}
    
\end{frame}

