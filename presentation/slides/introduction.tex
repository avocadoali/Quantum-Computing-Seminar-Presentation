\section{Einführung}

\subsubsection{Quanten Algorithmen}

    \begin{frame}
        \frametitle{Quanten Algorithmen}

        Wir haben schon viel über die wichtigsten Algorithmen gehört
        \begin{itemize}
            \item  Shors-Algorithmus
            \item  Grover-Algorithmus
        \end{itemize}

        \hfill

        Der HHL-Algorithmus
        \begin{itemize}
            \item  erstellt von Aram Harrow, Avinatan Hassidim und Seht Lloyd 
            \item  lösen von sehr großen linearen Gleichungen 
        \end{itemize}

        $$ A \vec{x} = \vec{b} $$

    \end{frame}


\subsubsection{Motivation}

    \begin{frame}
        \frametitle{Motivation}

        Es löst grundlegendes Probleme in der Mathematik
        \begin{itemize}
            \item   Least square fitting 
            \item   Optimierungs Probleme
            \item   Simulationen und Imageprocessing
            \item   ...
       \end{itemize}

        \hfil

        Kleine Revolution insbesondere bei Quantum Machine Learning
        \begin{itemize}
            \item HHL als Subroutine oder in erweiterten Form benutzt
            \item Abschätzung mit Computern brauchen mind N Zeitschritte!
       \end{itemize}


    \end{frame}

\subsubsection{Das Problem}
    \begin{frame}
        \frametitle{Das Problem}

        Gegeben: 
        \begin{itemize}
            \item $A$ Matrix der Form $n \times n$
            \item $\vec{b}$
       \end{itemize}

       \hfill

       Löse das System
        $$A \vec{x} = \vec{b}$$

    

        HHL verspricht uns einen exponentiellen Speedup!!

    \end{frame}


        
