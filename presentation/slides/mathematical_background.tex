\section{Mathematische Grundlagen}

\begin{frame}
    \frametitle{Hermitsche Matrix}
    \textbf{Sei: }
    \begin{itemize}
        \item   $A$ eine $n \times n$ Matrix
        \item   $A^T$ das transponierte von $A$
        \item   $\overline A$  das komplex konjugierter von $A$
        \item   $A^\dagger$ die Hermitsche Matrix von A
   \end{itemize}

   \hfil

    \textbf{Dann:}
    $$A = \overline {A^T} = A^\dagger $$
\end{frame}

\begin{frame}
    \frametitle{Hermitsche Matrix}
    
    \textbf{Beispiel:}
    $$A=\begin{bmatrix}2 & 1-i \\ 1+i & 3 \\ \end{bmatrix}$$
    $$\overline A = \begin{bmatrix} 2 & 1+i \\ 1-i & 3 \end{bmatrix}$$
    $$\overline {A^T} = \begin{bmatrix} 2 & 1-i \\ 1+i & 3 \\ \end{bmatrix}= A = A^\dagger$$

    \hfil

    \hfil

    Die Matrix $A$ ist Hermitisch.
\end{frame}

\begin{frame}
    \frametitle{Hermitsche Matrix}
    
    Falls eines Matrix $A$ nicht Hermitisch ist:

    \hfil

    $$A^\dagger = \begin{pmatrix} 0 & A \\ \overline {A^T} & 0 \end{pmatrix}$$
\end{frame}


\begin{frame}
    \frametitle{Spektralzerlegung}
    \textbf{Sei: }
    \begin{itemize}
        \item   $A$ eine $n \times n$ Matrix
        \item   $D$ ist eine Diagonalmatrix aus den Eigenwerten
        \item   $U$ besteht aus den Eigenvektoren von A
   \end{itemize}

    \hfil

    $$A =  U D U^T$$

    $$= \begin{bmatrix} U_1&U_2&...&U_n \end{bmatrix}
    \begin{bmatrix} \lambda_1 & 0 & 0 & 0\\ 0 & \lambda_2 &0 & 0\\ 0 & 0 & ... & 0\\ 0 & 0 & 0& \lambda_n \\ \end{bmatrix}
    \begin{bmatrix} U_1\\ U_2\\ ...\\ U_n\end{bmatrix}$$


\end{frame}

\begin{frame}
    \frametitle{Spektralzerlegung}
    Das Inverse von $A$ kann man folgendermaßen berechen:

    $$A^{-1}=  U^T D^{-1} U$$

    $$=\begin{bmatrix} U_1\\ U_2\\ ...\\ U_n\end{bmatrix} 
    \begin{bmatrix} \lambda_1^{-1} & 0 & 0 & 0\\ 0 & \lambda_2^{-1} &0 & 0\\ 0 & 0 & ... & 0\\ 0 & 0 & 0& \lambda_n^{-1} \end{bmatrix} 
    \begin{bmatrix} U_1&U_2&...&U_n \end{bmatrix}$$

 $$=\begin{bmatrix} U_1 \lambda_1^{-1} \\ U_2 \lambda_2^{-1}\\ ...\\ U_n \lambda_n^{-1} \end{bmatrix} 
    \begin{bmatrix} U_1&U_2&...&U_n \end{bmatrix}$$

    \hfil

    \begin{itemize}
        \item  $A^{-1}$ nur durch Eigenwerten und Eigenvektoren bestimmbar!
        \item  Methode im klassischen nicht schneller
        \item  für HHL Algorithmus sehr wichtig
   \end{itemize}
\end{frame}

\begin{frame}
    \frametitle{Veschränkung}
    
    Verschränkte Zustände können nicht durch einzelne Zustände dargestellt werden

    \hfil

    $$\ket{\Phi} \neq \ket{\phi} \ket{\psi}$$

\end{frame}

\begin{frame}
    \frametitle{Veschränkung}
    \textbf{Beispiel:}

    \hfil

    \begin{columns}[c]
        \begin{column}{0.6\hsize}\centering
        Nicht Verschränkt
        $$\ket{\Phi_1} = \frac{1}{\sqrt{2}} ( \ket{10} + \ket{11})$$
        $$= \ket{1}\otimes \frac{1}{\sqrt{2}} ( \ket{0} + \ket{1})= \ket{1} \ket{+}$$
        \end{column}

        \begin{column}{0.4\hsize}
        Verschränkt
        $$\ket{\Phi_2} = \frac{1}{\sqrt{2}} ( \ket{00} + \ket{11})$$
        $$\neq \ket{\alpha}\ket{\beta}$$

        \end{column}
    \end{columns}

\end{frame}